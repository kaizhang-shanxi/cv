\documentclass[11pt,a4paper,sans]{moderncv}
\moderncvstyle{casual}
\moderncvcolor{blue}
% 字体
\usepackage{xltxtra,fontspec,xunicode}
\usepackage[slantfont,boldfont,CJKchecksingle]{xeCJK}
\CJKsetecglue{\hskip 0.15em plus 0.05 em minus 0.05em}
\XeTeXlinebreaklocale "zh"
\XeTeXlinebreakskip=0pt plus 1pt minus 0.1pt
% \setCJKmainfont[BoldFont=Adobe Heiti Std,Scale=0.8]{Adobe Song Std}
\setCJKmainfont[BoldFont=Adobe Heiti Std]{Adobe Song Std}
\setCJKsansfont{Adobe Heiti Std}
\setCJKmonofont{WenQuanYi Micro Hei Mono}
\setCJKfamilyfont{FZLiuKai}{FZSuXinShiLiuKaiS-R-GB}
\newcommand{\FZLiuKai}{\CJKfamily{FZLiuKai}}
\renewcommand*{\namefont}{\fontsize{26}{28}\FZLiuKai}
\renewcommand*{\titlefont}{\fontsize{16}{18}}
\renewcommand*{\sectionfont}{\fontsize{12}{14}}
% 页面设置
\usepackage[scale=0.78]{geometry}
% \renewcommand{\baselinestretch}{0.9}\normalsize
% timeline
\usepackage[firstyear=2009,lastyear=2017]{moderntimeline}
\tlwidth{0.8ex}
\tltext{\tiny}
% 水平列表
\usepackage[inline]{enumitem}
% 个人信息
\name{张}{凯}
% \title{个人简历}
\address{海淀区清华大学26\#楼516室}{北京(邮编:100084)}
\phone[mobile]{+86-131 6174 3225}
\email{bibaijin@gmail.com}
% \homepage{bibaijin.github.io/homepage/}
\photo[56pt][0.2pt]{img/avatar}
%%%%%%%%%%%%%%%%%%%%
% 内容
%%%%%%%%%%%%%%%%%%%%
\begin{document}
\maketitle
\vspace{-2.5em}
\section{教育背景}
\tlcventry{2013/9}{0}{硕士}{清华大学}{北京}{\textit{GPA:3.1/4.0}\hfill 籍贯%
\hspace{0.3cm}山西忻州}{电子工程系,电子科学与技术\hfill 出生年月%
\hspace{0.3cm}1991 年 4 月}
\tlcventry{2009/9}{2013/7}{双学士}{中国科学技术大学}{合
肥}{\textit{GPA:3.5/4.3}}{电子工程与信息科学系,电子信息工程\hfill\textbf{双学
位}\hspace{0.3cm}\textbf{工商管理}}
\section{技能水平}
\cvitem{计算机}{熟悉 \textbf{Linux} 环境,熟悉 Shell、Python 和 Golang\hfill
英语\hspace{0.3cm}524 (六级)}
\section{项目经历}
\tldatecventry{2015/6}{后台开发实习生}{宜信大数据}{北京}
{\textbf{LAIN --- 云计算平台}}{
  %Lain 是一个 docker 管理系统,是一个 PaaS 平台。我完成的工作有:
  \begin{itemize*}[itemjoin=\hfill]
    \item 用 go 语言实现 proxy
    \item 用 python 写测试脚本,并用 Jenkins 集成
  \end{itemize*}}
\tlcventry{2015/7}{0}{维护者与编写者}{实验室}{北京}{电子系信息管理系统}{
  \begin{itemize*}[itemjoin=\hfill]
    \item 添加博士生论坛管理模块
    \item 重构数据库
    \item 重新实现登录等功能,重写界面
  \end{itemize*}}
\tldatecventry{2015/10}{编写者}{实验室}{北京}{社区发现}{
  \begin{itemize*}[itemjoin=\hfill]
    \item 知乎爬虫
    \item 局部社区发现
    \item 用回位随机游走进行 Top-k 推荐
  \end{itemize*}}
\tlcventry{2014/11}{0}{参与者}{实验室}{北京}{旋进波 --- 多路复用系统}
{%利用旋进波实现了一套多路复用系统。我完成的工作有:
  \begin{itemize*}[itemjoin=\hfill]
    \item 用 VB 实现计算机、发射机和接收机之间的通信
    \item 用 Matlab 解调出旋进波的模式
  \end{itemize*}}
\section{社工经历}
\tlcventry{2013/9}{2014/7}{党支部书记}{无研132}{北京}{班级管理}{
  \begin{itemize*}[itemjoin=\hfill]
    \item 领导组织班级活动,形成凝聚力
    \item 带领全班获得集体评优\textbf{全校第三名}的好成绩
  \end{itemize*}}
\tlcventry{2015/9}{0}{带班助理}{电子系研工组}{北京}{班级管理}{
  \begin{itemize*}[itemjoin=\hfill]
    \item 指导班级活动
    \item 参与筹划和组织电子系的重要活动
    \item 关心帮助有困难同学
  \end{itemize*}}
\tlcventry{2014/9}{2015/7}{参与者}{亚洲青年中心}{北京}{\textbf{研究生领导力训
练营}}{
  \begin{itemize*}[itemjoin=\hfill]
    \item 与留学生一起住宿,互相学习
    \item 参与组织圣诞节、保龄球赛和生日 party 等活动
  \end{itemize*}}
\section{曾获奖励}
\tldatecventry{2014/9}{清华大学``优秀研究生党支部书记''}{}{}{}{}
% \vspace{-0.8em}
\tldatecventry{2013/6}{中科大``校优秀毕业生''}{}{}{}{}
% \vspace{-0.8em}
\tlcventry{2010}{2012}{中科大``校优秀奖学金''}{铜奖}{三次}{}{}{}
\section{其他情况}
\cvitem{个人兴趣}{骑行;乒乓球,羽毛球;读书\hfill 自我评价\hspace{0.3cm}数理
基础好,兴趣广泛,踏实认真}
\clearpage
\end{document}
